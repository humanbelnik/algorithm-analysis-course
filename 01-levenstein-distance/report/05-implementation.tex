\chapter{Технологический раздел}

Раздел содержит описание средств реализации программы, листинги кода алгоритмов и функциональные тесты.

\section{Средства реализации}

Для реализации программы выбран язык программирования \textit{Python}~\cite{pydoc}. Данный язык поддерживает кодировку символов \textit{UTF-8} и позволяет замерить процессорное время работы алгоритма с помощью функции \textit{process\_time} модуля \textit{time}~\cite{pytimedoc}.


\section{Реализация алгоритмов}

Листинги~\ref{lst:levenstein.py},~\ref{lst:damerau.py},~\ref{lst:lev-rec.py},~\ref{lst:damerau-memo.py} содержат реализации алгоритмов Левенштейна и Дамерау~---~Левенштейна. Листинг~\ref{lst:init_matrix.py} содержит реализацию вспомогательной функции инициализации матрицы.

\includelisting
{levenstein.py} % Имя файла с расширением (файл должен быть расположен в директории inc/lst/)
{Алгоритм Левенштейна} % Подпись листинга

\clearpage

\includelisting
{damerau.py} % Имя файла с расширением (файл должен быть расположен в директории inc/lst/)
{Алгоритм Дамерау~---~Левенштейна} % Подпись листинга

\clearpage

\includelisting
{lev-rec.py} % Имя файла с расширением (файл должен быть расположен в директории inc/lst/)
{Рекурсивный алгоритм Левенштейна} % Подпись листинга

\clearpage

\includelisting
{damerau-memo.py} % Имя файла с расширением (файл должен быть расположен в директории inc/lst/)
{Рекурсивный алгоритм Дамерау~---~Левенштейна с мемоизацией} % Подпись листинга

\clearpage

\includelisting
{init_matrix.py} % Имя файла с расширением (файл должен быть расположен в директории inc/lst/)
{Алгоритм инициализации матрицы} % Подпись листинга

\clearpage

\section{Функциональное тестирование}

Таблица~\ref{tbl:test} содержит информацию о проведенном функциональном тестировании реализованных алгоритмов. Все тесты пройдены успешно. 

\begin{table}[ht]
	\small
	\begin{center}
			\caption{Функциональные тесты}
			\label{tbl:test}
			\begin{tabular}{|c|c|r|r|r|r|}
				\hline
				\multicolumn{2}{|c|}{\bfseries Входные данные}
				& \multicolumn{4}{c|}{\bfseries Расстояние и алгоритм} \\ 
				\hline 
				&
				& \multicolumn{1}{c|}{\bfseries Итер. Л} 
				& \multicolumn{1}{c|}{\bfseries Итер. Д-Л} 
				& \multicolumn{1}{c|}{\bfseries Рекур. Л} 
				& \multicolumn{1}{c|}{\bfseries Рекур. с мемо Д-Л} \\
				\hline
				" " & " " & 0 & 0 & 0 & 0 \\
				\hline
				" " & "qwe" & 3 & 3 & 3 & 3 \\
    			\hline
				"qwe" & " " & 3 & 3 & 3 & 3 \\
    			\hline
				"qweR" & "QWER" & 3 & 3 & 3 & 3 \\
    			\hline
				"qwerty" & "qwertyuiop" & 4 & 4 & 4 & 4 \\
    			\hline
				"qwer" & "qwre" & 2 & 1 & 2 & 1 \\
    			\hline
				"qwer" & "qare" & 3 & 2 & 3 & 2 \\
    			\hline
				"hello" & "привет" & 6 & 6 & 6 & 6 \\
    			\hline
				"45" & "-45" & 1 & 1 & 1 & 1 \\
    			\hline
				"qwer ber" & "qwer" & 4 & 4 & 4 & 4 \\
    			\hline
			\end{tabular}	
	\end{center}
\end{table}

\section*{Вывод}

В разделе были описаны средства реализации алгоритмов, приведены листинги кода и описание функционального тестирования.