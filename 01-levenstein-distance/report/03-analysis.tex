\chapter{Аналитический раздел}

\section{Алгоритм Левенштейна}
Расстояние Левенштейна — это число, характеризующее минимальное количество операций вставки, удаления или замены, необходимых для приведения двух строк к равенству~\cite{lev-def}. Для операций вставки, удаления или замены символа устанавливается единичная стоимость, а для операции сравнения двух символов — стоимость равна нулю.

Для определения расстояния Левенштейна между подстроками $S_1[1...i]$ и $S_2[1...j]$ вводится функция $D(i,j)$, которая определяется соотношением~(\ref{eq:L}):
\begin{equation}
	\label{eq:L}
	D(i, j) =
	\begin{cases}
		0, &\text{если } i = 0 \text{ и } j = 0, \\
		i, &\text{если } j = 0 \text{ и } i > 0, \\
		j, &\text{если } i = 0 \text{ и } j > 0, \\
		\min \begin{cases}
			D(i, j - 1) + 1, \\
			D(i - 1, j) + 1, \\
			D(i - 1, j - 1) +  \delta(S_{1}[i], S_{2}[j]),
		\end{cases}
		& \text{если } i > 0 \text{ и } j > 0.
	\end{cases}
\end{equation}

Здесь $\delta(S_1[i], S_2[j])$ определяется соотношением~(\ref{eq:delta}):
\begin{equation}
	\label{eq:delta}
	\delta(S_1[i], S_2[j]) = \begin{cases}
		0, & \text{если } S_1[i] = S_2[j], \\
		1, & \text{иначе}.
	\end{cases}
\end{equation}

Расстояние Левенштейна между строками $S_1$ и $S_2$ равно значению функции $D(L_1, L_2)$, где $L_1$ и $L_2$ — длины строк $S_1$ и $S_2$ соответственно. Для хранения значений функции $D(i,j)$ используется матрица размером $M \times N$, где $M$ — длина первой строки, а $N$ — длина второй строки.

\section{Алгоритм Дамерау~---~Левенштейна}
Дамерау модифицировал понятие редакционного расстояния Левенштейна. Он ввел операцию транспозиции двух символов и назначил ей единичную стоимость. Обновленный вид функции $D(i, j)$ имеет вид, представленный формулой~(\ref{eq:DL}): 
\begin{equation}
	\label{eq:DL}
	D(i, j) =
\begin{cases}
    0, & \text{} i = 0, j = 0 \\
    i, & \text{ } j = 0, i > 0 \\
    j, & \text{ } i = 0, j > 0 \\
    \min \begin{cases}
        D(i, j - 1) + 1, \\
        D(i - 1, j) + 1, \\
        D(i - 1, j - 1) + \delta(S_1[i],
     S_2[j])
    \end{cases}, & \parbox[t]{.55\linewidth}{\raggedright  $i > 0$, $j > 0$} \\
    \min \begin{cases}
        D(i, j), \\
        D(i - 2, j - 2) + 1
    \end{cases}, & \parbox[t]{.55\linewidth}{\raggedright  $S_1[i - 1] = S_2[j - 2]$,\\
        $S_1[i - 2] = S_2[j - 1]$}
\end{cases}
\end{equation}
где сравнение символов строк $\delta(S_1[i], S_2[j])$ задается как:
\begin{equation}
	\label{eq:delta}
	\delta(S_1[i], S_2[j]) = \begin{cases}
		0, & \text{если } S_1[i] = S_2[j] \\
		1, & \text{иначе}
	\end{cases}
\end{equation}

\section{Рекурсивный алгоритм Левенштейна}
Алгоритм Левенштейна можно реализовать рекурсивно, непосредственно используя формулу~(\ref{eq:L}). При этом нет необходимости хранить матрицу значений функции $D(i,j)$, но важно правильно задать условие выхода из рекурсии, чтобы избежать переполнения стека.

\section{Алгоритм Дамерау~---~Левенштейна с кешем}
В процессе вычисления значений функции $D(i,j)$ по формуле~(\ref{eq:DL}) результат для конкретной пары $(i, j)$ может быть вычислен несколько раз. Мемоизация вводится для уменьшения количества повторных вычислений. Для мемоизации используется структура данных, например, хеш-таблица, хранящая значения функции $D$ для конкретных пар $(i, j)$. Если для очередной пары значение уже вычислено, то оно берется из структуры, а не пересчитывается.

\section*{Вывод}

В данном разделе были описаны понятия расстояний Левенштейна и Дамерау~---~Левенштейна, приведены соотношения для вычисления соответствующих расстояний, а также рассмотрены возможные оптимизации алгоритмов.
