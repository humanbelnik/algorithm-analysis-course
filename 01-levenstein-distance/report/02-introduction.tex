\chapter*{ВВЕДЕНИЕ}
\addcontentsline{toc}{chapter}{ВВЕДЕНИЕ}

Задача определения редакционного расстояния между двумя символьными строками является актуальной. Соответствующие решения применяются в поисковых системах для обнаружения опечаток в набранном тексте, а также в биоинформатике для определения сходства между последовательностями ДНК.

Цель работы — описать, реализовать и сравнить алгоритмы вычисления редакционного расстояния Левенштейна и Дамерау~---~Левенштейна между двумя символьными строками.

Для достижения цели необходимо выполнить следующие задачи:
\begin{itemize}[label=--]
    \item описать вариации алгоритмов Левенштейна и Дамерау~---~Левенштейна;
    \item спроектировать и реализовать описанные вариации алгоритмов;
    \item для каждого алгоритма провести замер процессорного времени. 
\end{itemize}
