\chapter{Конструкторский раздел}

В данном разделе приведены схемы различных реализаций алгоритмов Левенштейна и Дамерау~---~Левенштейна.

\section{Алгоритм Левенштейна}
\includeimage
{levenstein} % Имя файла без расширения (файл должен быть расположен в директории inc/img/)
{f} % Обтекание (без обтекания)
{h} % Положение рисунка (см. figure из пакета float)
{1 \textwidth} % Ширина рисунка
{Схема нерекурсивного алгоритма Левенштейна} % Подпись рисунка

\clearpage

\section{Нерекурсивный Дамерау~---~Левенштейн}
\includeimage
{dam-levenstein} % Имя файла без расширения (файл должен быть расположен в директории inc/img/)
{f} % Обтекание (без обтекания)
{h} % Положение рисунка (см. figure из пакета float)
{1 \textwidth} % Ширина рисунка
{Схема нерекурсивного алгоритма Дамерау~---~Левенштейна} % Подпись рисунка

\clearpage

\section{Рекурсивный алгоритм Левенштейна}

На рисунке~\ref{img:recursive-levenstein}
приведена схема рекурсивной реализации алгоритма Левенштейна.

\includeimage
{recursive-levenstein} % Имя файла без расширения (файл должен быть расположен в директории inc/img/)
{f} % Обтекание (без обтекания)
{h} % Положение рисунка (см. figure из пакета float)
{1.0\textwidth} % Ширина рисунка
{Схема рекурсивного алгоритма Левенштейна} % Подпись рисунка

\clearpage

\section{Алгоритм Дамерау~---~Левенштейна с кешем}

\includeimage
{cache-dlr} % Имя файла без расширения (файл должен быть расположен в директории inc/img/)
{f} % Обтекание (без обтекания)
{h} % Положение рисунка (см. figure из пакета float)
{0.91\textwidth} % Ширина рисунка
{Схема рекурсивного алгоритма Дамерау~---~Левенштейна c мемоизацией, первая часть} % Подпись рисунка

\includeimage
{damerau-rec} % Имя файла без расширения (файл должен быть расположен в директории inc/img/)
{f} % Обтекание (без обтекания)
{h} % Положение рисунка (см. figure из пакета float)
{0.9\textwidth} % Ширина рисунка
{Схема рекурсивного алгоритма Дамерау~---~Левенштейна c мемоизацией, вторая часть} % Подпись рисунка

\clearpage

\section{Алгоритм инициализации матрицы}

\includeimage
{matrix} % Имя файла без расширения (файл должен быть расположен в директории inc/img/)
{f} % Обтекание (без обтекания)
{h} % Положение рисунка (см. figure из пакета float)
{0.6\textwidth} % Ширина рисунка
{Схема алгоритма инициализации матрицы значений функции} % Подпись рисунка


\section*{Вывод}

В данном разделе были приведены схемы различных алгоритмов Левенштейна и Дамерау~---~Левенштейна.

