\chapter{Пример работы программы}

Листинг~\ref{lst:cli.py} демонстрирует строку запуска программы с указанием входных данных в качестве аргументов командной строки и вывод программы в ходе работы: повторную печать параметров и информацию о количестве загруженных страниц.

\includelisting
{cli.py} % Имя файла с расширением (файл должен быть расположен в директории inc/lst/)
{Работа программы} % Подпись листинга

На рисунке~\ref{img:real} показан фрагмент реальной страницы ресурса, а на рисунке~\ref{img:downloaded} фрагмент загруженного HTML файла.

\includeimage
{real}
{f}
{h}
{1 \textwidth}
{Фрагмент страницы интернет ресурса}

\clearpage

\includeimage
{downloaded}
{f}
{h}
{1 \textwidth}
{Фрагмент загруженного HTML}

\clearpage

\chapter{Тестирование}

В таблице~\ref{tbl:test} представлены результаты функционального тестирования программы.

\begin{table}[ht]
    \small
    \begin{center}
        \begin{threeparttable}
            \caption{Результаты функционального тестирования программы}
            \label{tbl:test}
            \begin{tabular}{|r|r|r|}
                \hline
                \textbf{Количество страниц} & \textbf{Количество потоков} & \textbf{Ожидаемый результат}  \\
                \hline
                1 & 1  & Загружена 1 страница \\
                \hline
                1  & 12  & Загружена 1 страница \\
                \hline
                50  & 1  & Загружено 50 страниц \\
                \hline
                50  & 12  & Загружено 50 страниц \\
                \hline
                1000  & 1  & Загружено 1000 страниц \\
                \hline
                1000  & 12  & Загружено 1000 страниц \\
                \hline
            \end{tabular}
        \end{threeparttable}
    \end{center}
\end{table}

Все тесты пройдены успешно.