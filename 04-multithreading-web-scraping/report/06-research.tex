\chapter{Исследование}

\section{Технические характеристики}

Замер скорости проводился на ЭВМ со следующими характеристиками:
\begin{itemize}[label=--]
    \item операционная система \textit{Windows Home 11};
    \item объем оперативной памяти 16 ГБ;
    \item процессор \textit{13th Gen Intel(R) Core(TM) i5-13500H 2.60 GHz}~\cite{processor}.
\end{itemize}

\section{Проведение исследования}

В замере времени исследовалась зависимость скорости загрузки страниц от количества потоков обработки. Максимальное количество страниц для загрузки в каждом замере равно 150.

В таблице~\ref{tbl:bench} приведены результаты замера.

\begin{table}[ht]
    \small
    \begin{center}
        \begin{threeparttable}
            \caption{Результаты замера}
            \label{tbl:bench}
            \begin{tabular}{|r|r|}
                \hline
                \textbf{Количество потоков (шт)} & \textbf{Скорость загрузки (шт/сек)}  \\
                \hline
                1 & 2.69 \\
                \hline
                2 & 5.19 \\
                \hline
                4 & 7.04 \\
                \hline
                8 & 10.94 \\
                \hline
                16 & 12.36 \\
                \hline
                32 & 15.86 \\
                \hline
                48 & 15.40 \\
                \hline
            \end{tabular}
        \end{threeparttable}
    \end{center}
\end{table}

\clearpage

На рисунке~\ref{img:final_bench} результаты отображены графически.

\includeimage
{final_bench}
{f}
{h}
{1 \textwidth}
{Зависимость скорости загрузки страниц от количества потоков обработки}

Использование 32 потоков на данной ЭВМ показывает скорость в 5 раз большую, нежели при обработке с одним потоком. При использовании большего числа потоков скорость снижается, что может быть обусловлено накладными затратами на создание потоков и переключение контекста выполнения.