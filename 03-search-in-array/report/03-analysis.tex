\chapter{Аналитический раздел}

Раздел содержит определение используемого в работе словаря и описание алгоритмов поиска в нем.

\section{Определение словаря}
\label{chap:dict}
В данной работе под словарем будем подразумевать массив элементов длины $N$. Массив -- упорядоченная коллекция, доступ к элементам которой осуществляется произвольным образом по их индексам.

\section{Линейный поиск}
\label{chap:linear_search}
Алгоритм линейного поиска предполагает последовательный перебор элементов словаря до тех пор, пока искомый не будет найден. Алгоритм не предполагает предварительной обработки элементов, а искомый элемент может располагаться на любой позиции, что требует перебора всех элементов в худшем случае и обеспечивает линейную временную сложность~\cite{book_skiena}.

\section{Бинарный поиск}
Алгоритм бинарного поиска требует упорядоченности элементов словаря по возрастанию. Пусть искомый элемент именуется как $x$. Из словаря выбирается опорный элемент $pivot$. Упорядоченность элементов гарантирует:
\begin{itemize}[label=--]
    \item $x > pivot$, то есть $x$ правее $pivot$ и дальнейшего внимание требует только часть словаря правее $pivot$;
    \item $x < pivot$, то есть $x$ левее $pivot$ и дальнейшего внимание требует только часть словаря левее $pivot$;
    \item $x == pivot$, то есть искомый элемент найден.
\end{itemize}
Учитывая, что в общем случае искомый элемент $x$ может оказаться в любой ячейке словаря равновероятно, в качестве опорного элемента $pivot$ принято выбирать средний на рассматриваемом интервале элемент словаря. При таком выборе опорного элемента на каждой итерации алгоритма размер словаря усекается вдвое, что обеспечивает логарифмическую временную сложность его работы~\cite{book_skiena}. 

\clearpage

Графическая интерпретация этапов работы алгоритма представлена на рисунке~\ref{img:binsearch.pdf}:

\includeimage
{binsearch.pdf} % Имя файла без расширения (файл должен быть расположен в директории inc/img/)
{f} % Обтекание (без обтекания)
{h} % Положение рисунка (см. figure из пакета float)
{0.9 \textwidth} % Ширина рисунка
{Иллюстрация шагов работы алгоритма бинарного поиска} % Подпись рисунка

\section*{Вывод}    

В разделе дано определение используемого в работе словаря и описаны алгоритмы линейного и бинарного поиска.