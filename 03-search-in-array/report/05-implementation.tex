\chapter{Технологический раздел}

Раздел содержит описание средств реализации программы, листинги кода алгоритмов и функциональные тесты.

\section{Средства реализации}

Для реализации программы выбран язык программирования \textit{Python}~\cite{pydoc}.

\section{Реализация алгоритмов}

Листинги~\ref{lst:linear.py} и~\ref{lst:binary.py} содержат реализации рассматриваемых алгоритмов поиска:

\includelisting
{linear.py} % Имя файла с расширением (файл должен быть расположен в директории inc/lst/)
{Алгоритм линейного поиска} % Подпись листинга

\clearpage

\includelisting
{binary.py} % Имя файла с расширением (файл должен быть расположен в директории inc/lst/)
{Алгоритм бинарного поиска} % Подпись листинга

\clearpage

\section{Функциональное тестирование}

В таблицах~\ref{tab:linear} и~\ref{tab:binary} приведены результаты функционального тестирования реализаций алгоритмов.

\subsection{Алгоритм линейного поиска}

\begin{table}[h!]
    \centering
    \caption{Описание функционального тестирования алгоритма линейного поиска}
    \begin{tabular}{|r|c|r|r|}
        \hline
        \textbf{Тест} & \textbf{Входные данные} & \textbf{Ожидаемый выход} & \textbf{Фактический выход} \\
        \hline
        1 & [1, 2, 3, 4, 5], 3 & 2 & 2 \\
        \hline
        2 & [1, 2, 3, 4, 5], 6 & None & None \\
        \hline
        3 & [ ], 1 & None & None \\
        \hline
        4 & [5, 4, 3, 2, 1], 1 & 4 & 4 \\
        \hline
    \end{tabular}
    \label{tab:linear}
\end{table}

\subsection{Алгоритм бинарного поиска}

\begin{table}[h!]
    \centering
    \caption{Описание функционального тестирования алгоритма бинарного поиска}
    \begin{tabular}{|r|c|r|r|}
        \hline
        \textbf{Тест} & \textbf{Входные данные} & \textbf{Ожидаемый выход} & \textbf{Фактический выход} \\
        \hline
        1 & [1, 2, 3, 4, 5], 3 & 2 & 2 \\
        \hline
        2 & [1, 2, 3, 4, 5], 6 & None & None \\
        \hline
        3 & [ ], 1 & None & None \\
        \hline
        4 & [5, 4, 3, 2, 1], 1 & 0 & 0 \\
        \hline
    \end{tabular}
    \label{tab:binary}
\end{table}

Все тесты пройдены успешно.

\section*{Вывод}

В разделе были описаны средства реализации алгоритмов, приведены листинги кода и описание функционального тестирования.
