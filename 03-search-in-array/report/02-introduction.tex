\chapter*{ВВЕДЕНИЕ}
\addcontentsline{toc}{chapter}{ВВЕДЕНИЕ}

Поиск данных в хранилище является ключевой задачей при разработке информационных систем. Обработка данных требует их предварительного поиска и извлечения. Скорость поиска данных критически влияет на скорость работы системы.

Цель работы — описать, реализовать и сравнить алгоритмы линейного и бинарного поиска в словаре.

Для достижения цели необходимо выполнить следующие задачи:
\begin{itemize}[label=--]
    \item описать алгоритмы;
    \item спроектировать, реализовать и протестировать алгоритмы;
    \item провести замер количества операций сравнения в ходе работы алгоритмов при различных положениях искомого элемента в словаре.
\end{itemize}
