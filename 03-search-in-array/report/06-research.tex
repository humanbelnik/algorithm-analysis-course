\chapter{Исследовательский раздел}

Раздел содержит описание замера зависимости числа сравнений для поиска элемента в словаре от позиции элемента.

\section{Исследование зависимости количества сравнений от позиции искомого элемента}

Для алгоритмов линейного и бинарного поиска был проведен замер количества необходимых для поиска элемента сравнений в зависимости от позиции элемента. Замер проводился 10 раз, в качестве итогового значения выбиралось среднее арифметическое. Результаты замера приведены на гистограммах~\ref{img:l_gist.pdf} и~\ref{img:b_gist.pdf}:

\includeimage
{l_gist.pdf} % Имя файла без расширения (файл должен быть расположен в директории inc/img/)
{f} % Обтекание (без обтекания)
{h} % Положение рисунка (см. figure из пакета float)
{1 \textwidth} % Ширина рисунка
{Зависимость числа сравнений от позиции элемента для алгоритма линейного поиска} % Подпись рисунка


\includeimage
{b_gist.pdf} % Имя файла без расширения (файл должен быть расположен в директории inc/img/)
{f} % Обтекание (без обтекания)
{h} % Положение рисунка (см. figure из пакета float)
{1 \textwidth} % Ширина рисунка
{Зависимость числа сравнений от позиции элемента для алгоритма бинарного поиска} % Подпись рисунка

\clearpage

В случае линейного поиска сравнений тем больше, чем дальше искомый элемент располагается от начала словаря. В лучшем случае алгоритм требует одного сравнения, если искомый элемент первый. В худшем случае алгоритм производит $N+1$ сравнений -- перебирает все элементы массива, но так и не находит искомый и выходит из цикла.

В случае бинарного поиска количество сравнений не превышает $log(n)$ -- это худший случай, когда элемент дальше всего отстоит от середины словаря. Лучший случай -- одно сравнение, когда искомый элемент является средним в словаре.

В среднем алгоритм бинарного поиска является более эффективным, хоть и требует предварительной сортировки элементов.

\section*{Вывод}

В разделе был описан замер количества сравнений при поиске элемента в словаре.