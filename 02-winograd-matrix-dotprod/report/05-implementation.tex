\chapter{Технологический раздел}

Раздел содержит описание средств реализации программы, листинги кода алгоритмов и функциональные тесты.

\section{Средства реализации}

Для реализации программы выбран язык программирования \textit{Python}~\cite{pydoc}. Данный язык позволяет замерить процессорное время работы алгоритма с помощью функции \textit{process\_time} модуля \textit{time}~\cite{pytimedoc}.

\section{Реализация алгоритмов}

Листинги~\ref{lst:std.py}~-~\ref{lst:wopt.py} содержат реализации рассматриваемых алгоритмов умножения матриц:

\includelisting
{std.py} % Имя файла с расширением (файл должен быть расположен в директории inc/lst/)
{Стандартный алгоритм} % Подпись листинга

\clearpage

\includelisting
{w.py} % Имя файла с расширением (файл должен быть расположен в директории inc/lst/)
{Алгоритм Винограда} % Подпись листинга

\clearpage

\includelisting
{wopt.py} % Имя файла с расширением (файл должен быть расположен в директории inc/lst/)
{Оптимизированный алгоритм Винограда} % Подпись листинга

\clearpage

\section{Функциональное тестирование}

Во время функционального тестирования реализаций алгоритмов были рассмотрены следующие случаи:
\begin{itemize}[label=--]
    \item перемножение матриц размера $1 \times 1$;
    \item перемножение квадратных матриц;
    \item перемножение неквадратных матриц.
\end{itemize}

\subsection*{Тест 1. Перемножение матриц размера $1 \times 1$.}

Входные данные.
\[
A = \begin{bmatrix} 2 \end{bmatrix}, \quad B = \begin{bmatrix} 3 \end{bmatrix}
\]

Результат.
\[
A \times B = \begin{bmatrix} 6 \end{bmatrix}
\]

\subsection*{Тест 2. Перемножение квадратных матриц.}

Входные данные.
\[
A = \begin{bmatrix} 1 & 2 \\ 3 & 4 \end{bmatrix}, \quad B = \begin{bmatrix} 5 & 6 \\ 7 & 8 \end{bmatrix}
\]

Результат:
\[
A \times B = \begin{bmatrix} 1 \cdot 5 + 2 \cdot 7 & 1 \cdot 6 + 2 \cdot 8 \\ 3 \cdot 5 + 4 \cdot 7 & 3 \cdot 6 + 4 \cdot 8 \end{bmatrix} = \begin{bmatrix} 19 & 22 \\ 43 & 50 \end{bmatrix}
\]

\subsection*{Тест 3. Перемножение неквадратных матриц.}

Входные данные.
\[
A = \begin{bmatrix} 1 & 2 & 3 \\ 4 & 5 & 6 \end{bmatrix}, \quad B = \begin{bmatrix} 7 & 8 \\ 9 & 10 \\ 11 & 12 \end{bmatrix}
\]

Результат.
\[
A \times B = \begin{bmatrix} 1 \cdot 7 + 2 \cdot 9 + 3 \cdot 11 & 1 \cdot 8 + 2 \cdot 10 + 3 \cdot 12 \\ 4 \cdot 7 + 5 \cdot 9 + 6 \cdot 11 & 4 \cdot 8 + 5 \cdot 10 + 6 \cdot 12 \end{bmatrix} = \begin{bmatrix} 58 & 64 \\ 139 & 154 \end{bmatrix}
\]

Все тесты пройдены успешно.

\section*{Вывод}

В разделе были описаны средства реализации алгоритмов, приведены листинги кода и описание функционального тестирования.
