\chapter{Конструкторский раздел}

В данном разделе приведены схемы алгоритмов умножения матриц, рассмотренных в предыдущем разделе. Также для каждого алгоритма проведена оценка трудоемкости в соответствии с определённой моделью вычислений.

\section{Схемы алгоритмов}

\subsection{Стандартный алгоритм умножения матриц}

\includeimage
{std}
{f}
{h}
{0.6 \textwidth}
{Схема алгоритма стандартного умножения матриц}

\clearpage

\subsection{Алгоритм Винограда}

\includeimage
{win-01}
{f}
{h}
{0.6 \textwidth}
{Схема алгоритма Винограда. Первая часть}

\clearpage

\includeimage
{win-02}
{f}
{h}
{0.8 \textwidth}
{Схема алгоритма Винограда. Вторая часть}

\clearpage

\subsection{Оптимизированный алгоритм Винограда}

\includeimage
{optwin}
{f}
{h}
{0.8 \textwidth}
{Схема оптимизированного алгоритма Винограда. Первая часть}

\clearpage

\includeimage
{optwin-02}
{f}
{h}
{0.8 \textwidth}
{Схема оптимизированного алгоритма Винограда. Вторая часть}

\clearpage

\section{Оценка трудоемкости}

\subsection{Модель вычислений}
\label{chap:cmp-model}
Операции, имеющие единичную стоимость:
\begin{equation}
    \label{eq:op-1}
    \begin{gathered}
        +, -, =, +=, -=, ==, !=, <, >, <=, >=, [], ++, {-}-,\\
        \&\&, >>, <<, ||, \&, |
    \end{gathered}
\end{equation}

Операции, имеющие двойную стоимость:
\begin{equation}
    \label{eq:op-2}
    *, /, \%, *=, /=, \%=, \text{len}()
\end{equation}

Трудоемкость условного оператора:
\begin{equation}
    \label{eq:if}
    f_{if} = f_{\text{условия}} + 
    \begin{cases}
        \text{min}(f_1, f_2), & \text{лучший случай}\\
        \text{max}(f_1, f_2), & \text{худший случай}
    \end{cases}
\end{equation}

Трудоемкость цикла:
\begin{equation}
    \label{eq:loop}
    \begin{gathered}
        f_{loop} = f_{\text{инициализация}} + f_{\text{сравнения}} + M_{\text{итераций}} \times (f_{\text{тело}} +\\
        + f_{\text{инкремент}} + f_{\text{сравнения}})
    \end{gathered}
\end{equation}

Дальнейший расчет трудоемкости будет производиться с использованием данной модели.

\subsection{Трудоемкость классического алгоритма}

Стоимость начальной инициализации рассчитывается по формуле~(\ref{eq:init-std}):
\begin{equation}
    \label{eq:init-std}
    \begin{gathered}
        f_{init} = (2 + 1) + (2 + 1 + 1) + (2 + 1 + 1) \\
        + 2 + n \times (2 + p)
    \end{gathered} 
\end{equation}

Стоимость основного цикла рассчитывается по формуле~(\ref{eq:main-std}):
\begin{equation}
    \label{eq:main-std}
    \begin{gathered}
        f_{main} = 2 + n \times (2 + p \times (2 + m \times (2 + 9)))
    \end{gathered} 
\end{equation}

Итоговая стоимость трудоемкости рассчитывается по формуле~(\ref{eq:sum-std}):
\begin{equation}
    \label{eq:sum-std}
    \begin{gathered}
        f = f_{main} + f_{init} = 15 + 4 \times n + 3 \times n \times p + 11 \times m \times n \times p  \\ 
        \approx 11 \times m \times n \times p = O(N^3)
    \end{gathered} 
\end{equation}

\subsection{Трудоемкость алгоритма Винограда}

Стоимость начальной инициализации рассчитывается по формуле~(\ref{eq:init-w}):
\begin{equation}
    \label{eq:init-w}
    \begin{gathered}
        f_{init} = (2 + 1) + (2 + 1 + 1) + (2 + 1 + 1) \\
        + 2 + n \times (2 + p) = 13 + n \times (2 + p)
    \end{gathered} 
\end{equation}

Стоимость инициализации вспомогательных массивов вычисляется по формулам~(\ref{eq:initrow-w}) и~(\ref{eq:initcol-w}):
\begin{equation}
    \label{eq:initrow-w}
    \begin{gathered}
        f_{row} = 2 + 2 \times n + 8.5 \times m \times n
    \end{gathered} 
\end{equation}

\begin{equation}
    \label{eq:initcol-w}
    \begin{gathered}
        f_{col} = 2 + 2 \times n + 8.5 \times m \times p
    \end{gathered} 
\end{equation}

Стоимость основного цикла рассчитывается по формуле~(\ref{eq:main-w}):
\begin{equation}
    \label{eq:main-w}
    \begin{gathered}
        f_{main} = 2 + 2 \times n + n \times p \times (9 + 15 \times m) \\
        = 2 + 2 \times n + 9 \times n \times p + 15 \times m \times n \times p
    \end{gathered} 
\end{equation}

Добавочная трудоемкость цикла рассчитывается по формуле~(\ref{eq:noteven-loop-w}):
\begin{equation}
    \label{eq:noteven-loop-w}
    \begin{gathered}
        f_{check} = 3 + 
        \begin{cases}
            0, & \text{чётная} \\
            2 + 2n + 15\times n \times p, & \text{иначе}
        \end{cases}
    \end{gathered}  
\end{equation}

Итоговая стоимость трудоемкости рассчитывается по формуле~(\ref{eq:sum-w}):
\begin{equation}
    \label{eq:sum-w}
    \begin{gathered}
        f = f_{init} + f_{row} + f_{col} + f_{main} + f_{check}  
    \end{gathered} 
\end{equation}

В лучшем случае трудоемкость алгоритма составляет:
\begin{equation}
    \label{eq:best-w}
    \begin{gathered}
        f = 13 + n \times (2 + p) + 4 + 4 \times n + 8.5 \times m \times n + 8.5 \times m \times p + \\
        2 + 2n + 9 \times n \times p + 15 \times m \times n \times p \approx 15 \times m \times n \times p = O(N^3)
    \end{gathered} 
\end{equation}

В худшем случае трудоемкость алгоритма составляет:
\begin{equation}
    \label{eq:worst-w}
    \begin{gathered}
        f = 13 + n \times (2 + p) + 4 + 4 \times n + 8.5 \times m \times n + 8.5 \times m \times p + \\
        2 + 2 \times n + 9 \times n \times p + 15 \times m \times n \\ \times p + 2 + 2 \times n + 15 \times n \times p \approx 15  \\\times m \times n \times p = O(N^3)
    \end{gathered} 
\end{equation}

\subsection{Трудоемкость оптимизированного алгоритма Винограда}

Стоимость начальной инициализации рассчитывается по формуле~(\ref{eq:init-w-opt}):
\begin{equation}
    \label{eq:init-w-opt}
    \begin{gathered}
        f_{init} = (2 + 1) + (2 + 1 + 1) + (2 + 1 + 1) \\
        + 2 + n \times (2 + p) = 13 + n \times (2 + p)
    \end{gathered} 
\end{equation}

Стоимость инициализации вспомогательных массивов вычисляется по формулам~(\ref{eq:initrow-w-opt}) и~(\ref{eq:initcol-w-opt}):
\begin{equation}
    \label{eq:initrow-w-opt}
    \begin{gathered}
        f_{row} = 2 - 2.5 \times n + 4.5 \times m \times n
    \end{gathered} 
\end{equation}

\begin{equation}
    \label{eq:initcol-w-opt}
    \begin{gathered}
        f_{col} = 2 - 2.5 \times n + 4.5 \times m \times p
    \end{gathered} 
\end{equation}

Стоимость основного цикла рассчитывается по формуле~(\ref{eq:main-w-opt}):
\begin{equation}
    \label{eq:main-w-opt}
    \begin{gathered}
        f_{main} = 2 + 2 \times n + 16.5 \times p \times n + 8.5 \times m \times n \times p
    \end{gathered} 
\end{equation}

Добавочная трудоемкость цикла рассчитывается по формуле~(\ref{eq:noteven-loop-w}):
\begin{equation}
    \label{eq:noteven-loop-w}
    \begin{gathered}
        f_{check} = 3 + 
        \begin{cases}
            0, & \text{чётная} \\
            2 + 2 \times n + 15 \times n \times p, & \text{иначе}
        \end{cases}
    \end{gathered}  
\end{equation}

Итоговая стоимость трудоемкости рассчитывается по формуле~(\ref{eq:sum-w-opt}):
\begin{equation}
    \label{eq:sum-w-opt}
    \begin{gathered}
        f = f_{init} + f_{row} + f_{col} + f_{main} + f_{check}  
    \end{gathered} 
\end{equation}

В лучшем случае трудоемкость алгоритма составляет:
\begin{equation}
    \label{eq:best-w}
    \begin{gathered}
        f = 4 - 5 \times n + 4.5 \times m \times n + 4.5 \times m \times p + 2 + 2 \times n + 16.5 \times p \times n + \\
        8.5 \times m \times n \times p \approx 8.5 \times m \times n \times p = O(n^3)
    \end{gathered} 
\end{equation}

В худшем случае трудоемкость алгоритма составляет:
\begin{equation}
    \label{eq:worst-w}
    \begin{gathered}
        f = 4 - 5 \times n + 4.5 \times m \times n + 4.5 \times m \times \\ p + 2 + 2 \times n + 16.5 \times p \times n + 8.5 \times m \times n \times p + \\
        2 + 2 \times n + 15 \times p \times n \approx 8.5 \times m \times n \times p = O(n^3)
    \end{gathered} 
\end{equation}

\section*{Вывод}

В разделе были приведены схемы алгоритмов умножения матриц, а также рассчитана их трудоемкость. В данной модели вычислений все три алгоритма имеют кубическую сложность, а самым маленьким коэффициентом обладает оптимизированный алгоритм Винограда.
