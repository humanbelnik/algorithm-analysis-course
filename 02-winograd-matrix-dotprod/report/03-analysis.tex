\chapter{Аналитический раздел}

В разделе дано определение матрицы и операции умножения двух матриц. Описаны математические основания алгоритмов умножения матриц.

\section{Матрица как математический объект}
\label{chap:matrix-def}
Матрица $M_{m \times n}$ — таблица из $m$ строк и $n$ столбцов, содержащая элементы некоторого множества. В общем случае элементы матрицы могут быть любыми, но для определённости будем рассматривать числовые матрицы. Элемент матрицы $a_{ij}$ находится в $i$-й строке и $j$-м столбце. Строки и столбцы матрицы можно рассматривать как векторы.

\section{Операция умножения матриц}

\label{chap:dp-def}
Пусть даны матрицы $A_{m \times n}$ и $B_{n \times q}$. Результатом выполнения операции умножения $A \cdot B$ является матрица $C_{m \times q}$, элементы которой определяются согласно формуле~(\ref{eq:dp-element}): 
\begin{equation}
    \label{eq:dp-element}
    c_{ij} = \sum_{k=1}^m a_{ik} \cdot b_{kj}
\end{equation}

Стоит отметить два обстоятельства:
\begin{itemize}[label=--]
    \item операция умножения матриц некоммутативна;
    \item количество столбцов первой матрицы должно совпадать с количеством строк второй матрицы.
\end{itemize}

Операция скалярного произведения векторов $V = (v_1, \ldots, v_n)$ и $W = (w_1, \ldots, w_n)$ описывается формулой~(\ref{eq:scp}):
\begin{equation}
    \label{eq:scp}
    V \cdot W = v_1 \cdot w_1 + \ldots + v_n \cdot w_n
\end{equation}

Тогда под элементом $c_{ij}$ в формуле~(\ref{eq:dp-element}) можно понимать результат скалярного произведения $i$-й вектора-строки первой матрицы $A$ на вектор-столбец второй матрицы $B$ под индексом $j$.

\section{Стандартный алгоритм умножения матриц}
\label{chap:std-alg}

Стандартный алгоритм умножения матриц целиком базируется на математическом определении данной операции~(\ref{chap:dp-def}).

\section{Алгоритм Винограда}
\label{chap:vin}
Проанализируем стандартный алгоритм~(\ref{chap:std-alg}) на наличие повторных вычислений. Даны векторы $V = (v_1, \ldots, v_4)$ и $W = (w_1, \ldots, w_4)$. Их скалярное произведение определено формулой~(\ref{eq:scp}), но его можно переписать несколько иначе:
\begin{equation}
	\label{eq:vinscal}
	\begin{gathered} 
		V \cdot W = (v_1 + w_2) \cdot (v_2 + w_1) + (v_3 + w_4) \cdot (v_4 + w_3) \\
		- (v_1 \cdot v_2) - (v_3 \cdot v_4) - (w_1 \cdot w_2) - (w_3 \cdot w_4)
	\end{gathered}
\end{equation}

Сгруппированные в скобки выражения можно вычислить заранее, таким образом при расчете значения очередного элемента потребуется сделать две операции умножения и пять операций сложения. В стандартном алгоритме для аналогичного расчета применяется четыре операции умножения и три операции сложения. Аппаратно операция сложения эффективнее умножения, потому данный алгоритм должен показывать производительность лучше, нежели стандартный алгоритм.

В случае нечетного размера матрицы следует произвести дополнительную операцию -- добавление произведений последних элементов соответствующих строк и столбцов.

\section{Оптимизированный алгоритм Винограда}

Существует множество программных оптимизаций алгоритма Винограда~(\ref{chap:vin}). В данной работе будут применены следующие изменения:
\begin{itemize}[label=--]
    \item инкрементировать счётчик наиболее вложенного цикла на два;
    \item для инкремента использовать оператор \textit{+=};
    \item при вычислении вспомогательных массивов использовать декремент.
\end{itemize}

\section*{Вывод}    

В разделе было определено понятие матрицы и операции умножения матриц. Описаны алгоритмы умножения матриц.
