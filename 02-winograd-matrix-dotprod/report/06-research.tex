\chapter{Исследовательский раздел}

Раздел содержит описание замера процессорного времени работы рассматриваемых алгоритмов умножения матриц.

\section{Замер процессорного времени}
Замер процессорного времени выполнен на ноутбуке \textit{HUAWEI MateBook 14 Core Ultra}. Его характеристики:
\begin{itemize}[label=--]
    \item процессор \textit{Intel® Core™ Ultra 5 processor 125H};
    \item объем оперативной памяти 16 ГБ;
    \item ОС \textit{Ubuntu 22.03}~\cite{laptop}.
\end{itemize}

Замер проведен с помощью функции \textit{process\_time} модуля \textit{time}. Исследовалась зависимость времени работы алгоритма умножения от линейного размера входных квадратных матриц. Замер для каждого размера производился $N = 10$ раз. В качестве результата выбиралось среднее арифметическое.

В таблице~\ref{tbl:time} приведены результаты замера процессорного времени работы алгоритмов.

\begin{table}[ht]
    \small
    \begin{center}
        \begin{threeparttable}
            \caption{Замер времени для матриц размером от 1 до 301}
            \label{tbl:time}
            \begin{tabular}{|r|r|r|r|}
                \hline
                \textbf{Размер матрицы} & \textbf{Стандартный} & \textbf{Виноград} & \textbf{Оптимизированный Виноград} \\
                \hline
                1   & 0.000008  & 0.000016  & 0.000011  \\
                11  & 0.000161  & 0.000197  & 0.000157  \\
                21  & 0.000976  & 0.001040  & 0.000943  \\
                31  & 0.003113  & 0.003058  & 0.002724  \\
                41  & 0.006913  & 0.007010  & 0.006168  \\
                51  & 0.013487  & 0.015133  & 0.011492  \\
                61  & 0.024077  & 0.027390  & 0.025145  \\
                71  & 0.043437  & 0.041829  & 0.040956  \\
                81  & 0.058066  & 0.062298  & 0.049888  \\
                91  & 0.112635  & 0.081608  & 0.068125  \\
                101 & 0.116776  & 0.122409  & 0.091142  \\
                111 & 0.148105  & 0.147729  & 0.118995  \\
                121 & 0.178187  & 0.208828  & 0.160765  \\
                131 & 0.242555  & 0.259179  & 0.226213  \\
                141 & 0.309648  & 0.293353  & 0.257039  \\
                151 & 0.390584  & 0.367361  & 0.330666  \\
                161 & 0.430135  & 0.462702  & 0.379085  \\
                171 & 0.557330  & 0.540833  & 0.534774  \\
                181 & 0.630058  & 0.676642  & 0.529143  \\
                191 & 0.774023  & 0.756067  & 0.651742  \\
                201 & 0.890761  & 0.949499  & 0.810395  \\
                211 & 1.069306  & 1.017448  & 0.883063  \\
                221 & 1.189870  & 1.168395  & 1.066466  \\
                231 & 1.339761  & 1.373136  & 1.262913  \\
                241 & 1.525564  & 1.586061  & 1.369485  \\
                251 & 1.704104  & 1.763292  & 1.499202  \\
                261 & 2.056993  & 1.901680  & 1.722462  \\
                271 & 2.195457  & 2.331817  & 1.945634  \\
                281 & 2.475677  & 2.594344  & 2.210080  \\
                291 & 2.760152  & 2.716592  & 2.394943  \\
                301 & 3.134695  & 3.099551  & 2.664289  \\
                \hline
            \end{tabular}
        \end{threeparttable}
    \end{center}
\end{table}

\clearpage

На рисунке~\ref{img:bench.png} табличные данные отображены графически.

\includeimage
{bench.png} % Имя файла без расширения (файл должен быть расположен в директории inc/img/)
{f} % Обтекание (без обтекания)
{h} % Положение рисунка (см. figure из пакета float)
{1 \textwidth} % Ширина рисунка
{Зависимость процессорного времени работы алгоритма от линейного размера матриц} % Подпись рисунка

\section*{Вывод}

Замер процессорного времени подтвердил теоретический расчет трудоемкости. Все рассмотренные алгоритмы имеют схожую эффективность. На линейных размерах матриц до 80 элементов разницей во времени работы алгоритмов можно пренебречь. На размерах больше 80 алгоритм Винограда демонстрирует выигрыш в 7--10~\% по сравнению с другими алгоритмами. Стандартный алгоритм и алгоритм Винограда показывают идентичные результаты на рассматриваемых размерах входных данных.
